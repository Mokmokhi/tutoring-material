\documentclass[12pt]{article}
\usepackage{ctex}
\usepackage[a4paper, total={7in,9in}]{geometry}

\begin{document}
    \begin{center}
        \textbf{Variation: Homework}
    \end{center}

    In this Homework, we are going to 解ve for a variational problem.

    Suppose you are a manager of a company. You are now finding a new proposal on the company's expenses, which partly varies directly with stock price $P$, partly varies directly with salary expense $S$ together with number of employees $N$, and partly constant as properties cost. From companies history, you have known that:\begin{itemize}
        \item when stock price is increased by \$10 with other factors fixed, the company's expenses will be increased by \$10000;
        \item if salary expense is \$20000 and there are 50 employees, the company's expenses will be \$300000 more than that of when salary expense is \$10000 and there are 70 employees, where the stock price unchanged;
        \item when stock price is \$150 and there are 100 employees with salary \$15000, the company's expenses will be \$2000000.
    \end{itemize}

    \begin{enumerate}
        \item Denote the company's expense by $E$, find $E$ in terms of $P,S$ and $N$.
        \item Assume the company has been running with fixed salary and fixed properties cost, with stock price unchanged for years. If the company hires 30 more employees with the same salary as before, what is the percentage change of the company's expense?
        \item Further assume that the stock price will always be within \$100 and \$200, and in order to stabilize the expense, the employer would like to adjust every factor according to stock price. He sees the salary expense $S$ should be proportional to $(40000-P^2)$, and number of employees $N$ should be proportional to $(200-P)$. This time when $P=100$, the company's expense is estimated to be \$3450000.\begin{enumerate}
            \item Find $E$ in terms of $P$.
            \item Find $E$ when $P=200$. Explain why this is not the case we have to study.
            \item When will the expense be \$1375000? Explain you answer briefly.[Hint: Compute $E$ for if $P=-100$]
        \end{enumerate}
    \end{enumerate}

    \newpage

    \begin{center}
        \textbf{功課:變分法}
    \end{center}

    在這份功課中,我們將會解決一條變分的題目。

    設你為一間公司的經理。現你正檢視一份公司縂開支計畫書,部分開支隨股票價格 $P$ 而正變 ,而另一部分隨人均薪資 $S$ 和員工人數 $N$ 而正變,一部分物業成本為不變常數。回顧公司發展,已知 :\begin{itemize}
        \item 當股票價格提升\$10 而其他開支不變,公司縂開支將上升 \$10000;
        \item 如人均薪酬為 \$20000 而員工人數為 50 人,設股票價格不變,那麽比起人均薪酬為 \$10000 而員工人數為 70 人,公司縂開支將提高\$300000;
        \item 當股票價格為\$150,并且雇用了 100 位人均薪酬\$15000 的員工,公司縂開支為 \$2000000.
    \end{itemize}

    \begin{enumerate}
        \item 設$E$為公司縂開支,以$P$、$S$及$N$表$E$。
        \item 設公司多年以不變股票價格、薪資和物業成本運作。如公司以同薪資僱用多 30 位員工,求公司縂開支的百分變化。
        \item 再設股票價格始終在\$100 與\$200 以內,為了穩定開支,雇主想要根據股票價格 調整每項因素。他認爲人均薪酬 $S$ 應與 $(40000-P^2 )$ 成正比,員工數量 $N$ 應與$(200-P)$成正比。當 $P=100$,公司縂開支估計為\$3450000。\begin{enumerate}
            \item 以$P$表$E$。
            \item 當 $P=200$ 求 $E$。說明為何這不是我們將研究的情況。
            \item 在何種情況下縂開支會為\$1375000?請簡略解釋你的答案。[提示:如$P=-100$,找 $E$]
        \end{enumerate}
    \end{enumerate}
\end{document}