\documentclass[12pt]{article}
\usepackage{ctex}
\usepackage[a4paper, total={7in,9in}]{geometry}

\begin{document}
    \begin{center}
        \textbf{Homework 1}
    \end{center}

    In this Homework, we are going to solve for a variational problem.

    Suppose you are a manager of a company. You are now finding a new proposal on the company's expenses, which partly varies directly with stock price $P$, partly varies directly with salary expense $S$ together with number of employees $N$, and partly constant as properties cost. From companies history, you have known that:\begin{itemize}
        \item when stock price is increased by \$10 with other factors fixed, the company's expenses will be increased by \$10000;
        \item if salary expense is \$20000 and there are 50 employees, the company's expenses will be \$300000 more than that of when salary expense is \$10000 and there are 70 employees, where the stock price unchanged;
        \item when stock price is \$150 and there are 100 employees with salary \$15000, the company's expenses will be \$2000000.
    \end{itemize}

    \begin{enumerate}
        \item Denote the company's expense by $E$, find $E$ in terms of $P,S$ and $N$.
        \item Assume the company has been running with fixed salary and fixed properties cost, with stock price unchanged for years. If the company hires 30 more employees with the same salary as before, what is the percentage change of the company's expense?
        \item Further assume that the stock price will always be within \$100 and \$200, and in order to stabilize the expense, the employer would like to adjust every factor according to stock price. He sees the salary expense $S$ should be proportional to $(40000-P^2)$, and number of employees $N$ should be proportional to $(200-P)$. This time when $P=100$, the company's expense is estimated to be \$3450000.\begin{enumerate}
            \item Find $E$ in terms of $P$.
            \item Find $E$ when $P=200$. Explain why this is not the case we have to study.
            \item When will the expense be \$1375000? Explain you answer briefly.[Hint: Compute $E$ for if $P=-400$]
        \end{enumerate}
    \end{enumerate}
\end{document}