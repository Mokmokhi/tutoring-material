\documentclass[12pt]{article}
\usepackage{ctex}
\usepackage{amsmath}
\usepackage[a4paper, total={7in,9in}]{geometry}

\begin{document}
    \begin{center}
        \textbf{Locus and Circle: Homework}
    \end{center}

    In this homework, we are going to 解ve a problem related to circle.
    
    Given two points $A(10,0), B(-10,0)$ on the rectangular coordinate plane.

    \begin{enumerate}
        \item Find the equation of the perpendicular bisector $L$ of the straight line $AB$, and the midpoint $M$ of $A$ and $B$. Show that $M$ is on $L$.
        \item Denote $P$ and $Q$ be points such that $P$ is equidistant from $A$ and $L$, and $Q$ is equidistant from $B$ and $L$. \begin{enumerate}
            \item Denote the locus of $P$ by $C_1$ and locus of $Q$ by $C_2$. Find the equation of $C_1$ and $C_2$.
            \item Denote the closest point on $C_1$ to $M$ by $X$, and that on $C_2$ to $M$ by $Y$. What is the geometric relationship between $A$, $B$, $X$, $Y$, $M$?
            \item Find the equation of circle with radius $XM$ and center $M$.
        \end{enumerate}
        \item Denote the circle in (2c) by $C_0$. Denote the intersection points of $C_0$ and $L$ by $H$ and $K$. \begin{enumerate}
            \item Find the equations of tangent to $C_0$ at $H$,$K$,$X$,$Y$.
            \item Find the area enclosed by the above tangent lines and $C_0$.
        \end{enumerate}
    \end{enumerate}

    \newpage

    \begin{center}
        \textbf{功課:圓形與軌跡}
    \end{center}

    在這份功課中,我們將會解決一條與圓相關的題目。
    
    已知$A(10,0)$和 $B(-10,0)$ 為直角坐標平面上的兩點。

    \begin{enumerate}
        \item 求直線$AB$的垂直平分線$L$的方程,及$AB$的中點$M$。證明$L$通過$M$。
        \item 設 $P$ 和 $Q$ 為兩點使得 $P$ 與 $A$ 和 $L$ 等距,$Q$ 與 $B$ 和 $L$ 等距。 \begin{enumerate}
            \item 設$C_1$ 為$P$的軌跡,設$C_2$ 為$Q$的軌跡。求$C_1$ 及$C_2$ 的方程式。
            \item 以$X$表$C_1$上與$M$的最接近點,以$Y$表$C_2$上與$M$的最接近點。求$A$、$B$、$X$、$Y$、$M$之間的幾何關係。
            \item 求半徑為$XM$,圓心為$M$的圓方程。
        \end{enumerate}
        \item 用$C_0$ 表(2c)描述的圓,並用$H$及$K$表示$C_0$ 和$L$的交點。 \begin{enumerate}
            \item 求$C_0$在$H$、$K$、$X$、$Y$處的切線方程。
            \item 求上述切線和$C_0$所圍成的面積。
        \end{enumerate}
    \end{enumerate}
\end{document}