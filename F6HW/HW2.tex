\documentclass[12pt]{article}
\usepackage{ctex}
\usepackage{amsmath}
\usepackage[a4paper, total={7in,9in}]{geometry}

\begin{document}
    \begin{center}
        \textbf{Linear programming: Homework}
    \end{center}

    In this homework, we are going to 解ve a linear programming problem.

    Continued from previous homework, we now have different constraints to the company. Recall that the total expense of the company is estimated by $E=1000P+NS+350000$, where $P$ is the stock price, $N$ is the number of employees and $S$ is the salary of each employee.

    \begin{enumerate}
        \item For the relation of $N$ and $S$:\begin{align*}
            \begin{cases}
                0\leq N\leq 200\\
                0\leq S\leq 40000\\
                40000\leq S+200N \leq 60000
            \end{cases}
        \end{align*}
        \begin{enumerate}
            \item Sketch the region for the system of inequalities.
            \item Find the minimum of $NS$.
        \end{enumerate}
        \item Now the proposal states that the company expenses is proposed to be $2000000$, and luckily it is approved by the employer. What will be the maximum residue (remaining budget) for the company in this proposal?
        \item If the company residue is around maximum, at most how many more employees could be hired using residue under corresponding salary?
    \end{enumerate}

    \newpage

    \begin{center}
        \textbf{功課:線性規劃}
    \end{center}

    在這份功課中,我們將會解決一條線性規劃的題目。

    承接上一份功課的情景,現在我們對公司有不同的限制。回顧公司總費用估計為 $E=1000P+NS+350000$,其中$P$是股票價格,$N$是員工數量,$S$是每個員工的工資。

    \begin{enumerate}
        \item $N$與$S$的關係:\begin{align*}
            \begin{cases}
                0\leq N\leq 200\\
                0\leq S\leq 40000\\
                40000\leq S+200N \leq 60000
            \end{cases}
        \end{align*}
        \begin{enumerate}
            \item 畫出不等式組的圖像,並以陰影區域表示。
            \item 求$NS$的最小值。
        \end{enumerate}
        \item 現設公司費用為$2000000$,並且通過雇主的批准。求該計畫中公司剩餘的最大預算是多少?
        \item 如公司剩餘預算近最大值,在相應的工資下,最多可再雇用多少名員工?
    \end{enumerate}
\end{document}