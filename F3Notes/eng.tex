\documentclass[12pt]{article}
\usepackage{ctex}
\usepackage{amsmath,amsthm,amssymb}
\usepackage{graphicx,float}
\usepackage{fancyhdr}
\usepackage{array}
\usepackage[a4paper, total={7in,9in}]{geometry}

\fancyhf{}
\fancyhf[HL]{Pre S.3 Summer course}
\fancyhf[FR]{P.\thepage}

\newtheorem*{theorem}{Theorem}
\newtheorem{example}{Example}
\renewenvironment{proof}{\textit{Proof. }}{\hfill\qedsymbol}
\newtheorem*{sol}{Solution}

\begin{document}
    \begin{center}
        \textbf{Introductory probability}
    \end{center}

    Do you know how high the chance of guessing a mark six result correct is? It is the time for you to understand the method of computing probability.
    
    \begin{center}
        \textbf{Difference between probability and possibility}
    \end{center}

    We always talk about possibility and probability together. In fact, they are slightly different from one another. We would like to state the difference between two terms.

    \textbf{Possibility} is a term to describe the existence of the event, meaning you have the possibility to get full marks in the exam, because the score ranges there and such event appears possible; while the term \textbf{probability} refers to some calculation on how often (or the tendency) for such event to appear, just like we all have a very low probability to become a trillionaire (億萬富豪).

    One concrete example to differentiate probability and possibility in your mind would be your mum always tells you that you have the chance (possibility) to get full marks in your exam, but you know the chance (probability) of getting full marks is nearly zero, when you are still struggling about passing the exam.

    \begin{center}
        \textbf{Step 1: Identifying target events}
    \end{center}

    To calculate a correct probability, we should first find out the target events we are looking for. We shall gain the insight by taking look at some examples:

    \begin{example}
        Getting odd number result when rolling a cubic fair die.
    \end{example}

    In this example, we know that a die consists of six faces, for each face we have either 1 to 6 dots, corresponds to the values 1 to 6 respectively. For getting odd number result, it is equivalent to saying having 1,3,5 as target, over all 6 events.

    \begin{example}
        Getting even number result when rolling a cubic fair die.
    \end{example}

    In this example, we know that a die consists of six faces, for each face we have either 1 to 6 dots, corresponds to the values 1 to 6 respectively. For getting even number result, it is equivalent to saying having 2,4,6 as target, over all 6 events.

    \begin{example}
        Drawing an Ace from a deck of shuffled pokers.
    \end{example}

    In this scene, we could not predict the position of the Aces, so it is the so called random case. Our target would be the 4 Aces over all 52 cards in a deck of pokers.

    \begin{example}
        Giving born to a girl.
    \end{example}

    Let us make our world ideal to consist of only two genders, boys and girls. Then our target event is girl.

    \begin{center}
        \textbf{Step 2: Thinking probability as portion of selectable objects}
    \end{center}

    Let us think of rolling a cubic fair die, having 6 evenly distributed faces so that all faces have the equal chance to be rolled. This case we know that each face has the probability $\dfrac{1}{6}$.
    
    If we include the concept of target event, it should be easy to find the following thinking:

    \begin{theorem}
        Let $P_Y(X)$ be the probability of having event $X$ in event $Y$. Denote $\#X$ the number of event $X$ and $\#Y$ the number of event $Y$. Then $$P_Y(X)=\frac{\#X}{\#Y}$$
    \end{theorem}

    \begin{example}
        Getting odd number result when rolling a cubic fair die.
    \end{example}

    \begin{align*}
        P_{(1,2,3,4,5,6)}((1,3,5))=\frac{\#(1,3,5)}{\#(1,2,3,4,5,6)}=\frac{3}{6}=\frac{1}{2}
    \end{align*}

    \begin{example}
        Getting even number result when rolling a cubic fair die.
    \end{example}

    \begin{align*}
        P_{(1,2,3,4,5,6)}((2,4,6))=\frac{\#(2,4,6)}{\#(1,2,3,4,5,6)}=\frac{3}{6}=\frac{1}{2}
    \end{align*}

    \begin{example}
        Drawing an Ace from a deck of shuffled pokers.
    \end{example}

    \begin{align*}
        P_{Pokers}(Ace)=\frac{\#(Ace)}{\#(Pokers)}=\frac{4}{52}=\frac{1}{13}
    \end{align*}

    \begin{example}
        Giving born to a girl.
    \end{example}

    \begin{align*}
        P_{Genders}(Girls)=\frac{\#Girls}{\#Genders}=\frac{1}{2}
    \end{align*}

    \begin{center}
        \textbf{More Examples on probability}
    \end{center}

    We will locate more probability examples to consolidate the thinking of target event with the help of different arithmetic techniques.

    \begin{example}
        Find the probability of getting a sum equal 7 when rolling two fair dice at the same time.
    \end{example}

    \begin{example}
        Find the probability of getting a product equal 12 when rolling two fair dice at the same time. 
    \end{example}

    \begin{example}
        Find the probability of getting first prize in mark six.
    \end{example}

    \begin{example}
        Find the probability of hitting the central region in a shooting board.
    \end{example}

    \begin{center}
        \textbf{Expected value}
    \end{center}
\end{document}